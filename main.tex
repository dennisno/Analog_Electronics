% ================================== HEADER ====================================
\documentclass{article}           % Sets style/look of many things. 
% \documentclass{report}          % part, chapters, front page etc.
\usepackage[utf8]{inputenc}       % Encoding of input files UTF-8
\usepackage[T1]{fontenc}
\usepackage[scaled]{beramono}     % Font
\usepackage{color}                % Color text
\usepackage{titlesec}             % Select alternative section titles
\usepackage{fancyvrb}
\usepackage{verbatim}             % Comment environment
\usepackage{listings}             % Format and render text/code etc.
\usepackage{minted}               % Much better syntax highlighting
\usepackage{float}                % Control of floating environment/figure
\usepackage{graphicx,  subfigure} % Better figures, graphics, units etc.
\usepackage{multicol}             % Multiple columns
\usepackage{amsmath}              % Math: Equation, split, align etc.
\numberwithin{equation}{subsection}
\usepackage{siunitx}              % SI units
\usepackage{mathtools}            % Different math tools to use with amsmath
\usepackage{amssymb}              % Math symbols
\usepackage{varioref}
\usepackage[
    colorlinks,
    citecolor=black,              % I like links with standard black color
    filecolor=black,
    linkcolor=black,
    urlcolor=black,
    hypertexnames=false
]{hyperref}                       % Links in TOC etc.
\usepackage{cleveref}
\usepackage[all]{hypcap}          % Better links to floating environment
\setlength\columnsep{24pt}
\usepackage{tabularx}
\setlength{\extrarowheight}{10pt}
\usepackage{graphicx}
\graphicspath{{figures/}{../figures/}}
\usepackage[
backend=biber,
sorting=none
]{biblatex}
\addbibresource{bibliography.bib}

\renewcommand{\thesubsubsection}{\thesubsection.\alph{subsubsection}}
\renewcommand{\theequation}{\thesubsection.\arabic{equation}}

\title{{\color{red} WIP} Introduction to Analog Electronics}
\author{Ole Herman Schumacher Elgesem}
\date{\today}

\addtolength{\oddsidemargin}{-1.2in}
\addtolength{\evensidemargin}{-1.2in}
\addtolength{\textwidth}{2.4in}

\addtolength{\topmargin}{-0.8in}
\addtolength{\textheight}{1.6in}


% Removing paragraph indents is sometimes useful:
\setlength\parindent{0pt}

% ==============================================================================

% ================================= EXAMPLES ===================================
\begin{comment}

% TABLE (TABULAR):
\begin{center}
\begin{table}[]
  \begin{tabular}{|l|c|c|c|}
    \hline
    Colors: & Red    & Green  & Blue \\ \hline
    Red     & Red    & Yellow & Purple \\ \hline
    Green   & Yellow & Green  & Cyan \\ \hline
    Blue    & Purple & Cyan   & Blue \\ \hline
  \end{tabular}
  \caption{Caption}
  \label{tab:my_label}
\end{table}
\end{center}

% FIGURE:
The plot in figure \ref{fig:unique_name} clearly shows something important.
\begin{figure}[H]
\begin{center}
\includegraphics[width=0.3\textwidth]{figs/silicon.png}
\caption{The caption explains the figure.}
\label{fig:unique_name}
\end{center}
\end{figure}

% EQUATION:
\begin{equation}
\frac{df}{dt} = \lim_{h \to 0}\frac{f(t+h)-f(t)}{h}
\end{equation}

% ALIGNED MATH:
\begin{align*}
    r      &= \sqrt{x^2 + y^2 + z^2} & x &= r\cos \theta \sin \phi \\
    \theta &= \tan^-1(\frac{y}{x})   & y &= r\sin \theta \sin \phi \\
    \phi   &= \cos^-1(\frac{z}{r})   & z &= r\cos \phi
\end{align*}

% CENTERED DISPLAY MODE MATH:
\[ V_{th} = 0.45V \]

% INLINE MATH:
We used a supply voltage of 1.2 volts; \(V_{dd} = 1.2\text{V}\).

% MATRIX:
\[
\begin{Bmatrix} % Curly brackets
v_1 \\
v_2
\end{Bmatrix}
= \begin{bmatrix} % Square brackets
Z_{11} & Z_{12} \\
Z_{21} & Z_{22}
\end{bmatrix}
\begin{Bmatrix}
i_1 \\
i_2 
\end{Bmatrix}
\]

% COLORS:
\definecolor{orange}{RGB}{255,127,0}
{\color{red} Red text.} {\color{orange} Orange text.}

% SOURCE CODE:
\inputminted{Python}{src/hello.py}

% MULTIPLE COLUMNS:
\begin{multicols}{2}
Write text and add figures etc. Content will be automatically split. You can 
put figures, tables and even other multicols. Multiple columns can reduce 
wasted space in a document.
\end{multicols}

% FOOTNOTES:
A simple footnote\footnote{Additional information} or 
footnotemark\footnotemark{} is more readable.
\footnotetext{Footnotes are usefule for definitions, clarifications, 
background info, e-mail/website info etc.}

\end{comment}
% ==============================================================================

% ================================= DOCUMENT ===================================
\begin{document}
\maketitle
\tableofcontents
\newpage

\section*{Introduction}\label{sec:intro} % unnumbered section: \section*
Background...
\newpage

\section{Formulae}\label{sec:formula}
\subsection{Definitions}
\begin{tabular}{| l | l | l |}
\hline
& Ohm, $\Omega$ & Siemens, S \\ \hline
Real(DC) & \textbf{Resistance}, R & \textbf{Conductance}, G \\
\hline
Imaginary(AC) & Capacitive and Inductive \textbf{Reactance} $X_C, X_L$ 
& \textbf{Susceptance}, B \\ 
\hline
Total(Combined)& \textbf{Impedance}, Z & \textbf{Admittance}, Y \\ 
\hline
\end{tabular}
\newline
\noindent
\begin{tabularx}{\textwidth}{@{} 
p{\dimexpr.24\linewidth-2\tabcolsep-1.3333\arrayrulewidth} 
p{\dimexpr.26\linewidth-2\tabcolsep-1.3333\arrayrulewidth} 
p{\dimexpr.25\linewidth-2\tabcolsep-1.3333\arrayrulewidth} 
p{\dimexpr.25\linewidth-2\tabcolsep-1.3333\arrayrulewidth} @{}}
Energy(Joule, J):           & $ E = F * d = \frac{kg*m^2}{s^2}$
& Power(Watt, W):           & $ P = \frac{E}{t} = \frac{kg*m^2}{s^3}$\\
Charge(Coloumb, C):         & $ Q = I * t = A*s $
& Current(Ampere, A):       & $ I = \frac{Q}{t} = A$\\
Voltage(Volt, V):           & $ V = \frac{P}{I} = \frac{kg*m^2}{A*s^3}$
& Capacitance(Farad, F):    & $ C = \frac{Q}{V} = \frac{A^2s^4}{kg*m^2}$\\
Resistance(Ohm, $\Omega$):  & $ R = \frac{V}{I} = \frac{kg*m^2}{A^2*s^3}$
& Conductance(Siemens, S):  & $ G = \frac{1}{R} = \frac{A^2*s^3}{kg*m^2}$\\
Frequency(Hertz, Hz):       & $ f = \frac{1}{T} = s^{-1}$
& Period(Second, s):        & $ T = \frac{1}{f} = s$\\
Inductance(Henry, H):       & $ L = \frac{N^2 \mu A}{I} = \frac{kg*m^2}{A^2*s^2}$
& Induction voltage:        & $ v(t) = L*\frac{\delta i}{\delta t} $\\
Voltage Gain:               & $ A = \frac{v_{out}}{v_{in}}$
& Current Gain:             & $ \beta = \frac{i_{out}}{i_{in}} $ 
\end{tabularx}

\subsection{Resistors}
\begin{tabularx}{\textwidth}{@{} 
p{\dimexpr.24\linewidth-2\tabcolsep-1.3333\arrayrulewidth} 
p{\dimexpr.26\linewidth-2\tabcolsep-1.3333\arrayrulewidth} 
p{\dimexpr.25\linewidth-2\tabcolsep-1.3333\arrayrulewidth} 
p{\dimexpr.25\linewidth-2\tabcolsep-1.3333\arrayrulewidth} @{}}
Resistors in Series:        & $ R_{TOT} = R_1 + R_2 + ... + R_n $
& $ \Rightarrow $           & $ G_{TOT} = \frac{1}{\frac{1}{G_1}+\frac{1}{G_2}+...+\frac{1}{G_n}} $ \\
Resistors in Parallel:     & $ G_{TOT} = G_1 + G_2 + ... + G_n $
& $ \Rightarrow $           & $ R_{TOT} = \frac{1}{\frac{1}{R_1}+\frac{1}{R_2}+...+\frac{1}{R_n}} $ \\
Voltage Divider:            & $ V_x = V_s\frac{R_x}{R_{TOT}} $
& & 
\end{tabularx}

\subsection{Capacitors and Inductors}
\noindent\begin{tabularx}{\textwidth}{@{} 
p{\dimexpr.24\linewidth-2\tabcolsep-1.3333\arrayrulewidth} 
p{\dimexpr.26\linewidth-2\tabcolsep-1.3333\arrayrulewidth} 
p{\dimexpr.25\linewidth-2\tabcolsep-1.3333\arrayrulewidth} 
p{\dimexpr.25\linewidth-2\tabcolsep-1.3333\arrayrulewidth} @{}}
Capacitors in parallel:    & $ C_{TOT} = C_1 + C_2 + ... + C_n $
& Capacitors in Series:     & $ C_{TOT} = \frac{1}{\frac{1}{C_1}+\frac{1}{C_2}+...+\frac{1}{C_n}} $ \\
Inductors in Series:        & $ L_{TOT} = L_1 + L_2 + ... + L_n $
& Inductors in parallel:   & $ L_{TOT} = \frac{1}{\frac{1}{L_1}+\frac{1}{L_2}+...+\frac{1}{L_n}} $ \\
Capacative Reactance:       & $ X_C = \frac{1}{2*\pi*f*C} $
& Inductive Reactance:      & $ X_L = 2\pi*f*L $ \\
Capacitor Current:          & $ I_C = \frac{\delta V}{\delta t} * C $
& &\\
\multicolumn{4}{@{} l @{}}{Capacitor charge/discharge: $ v(t) = V_F + (V_I - V_F) * e^\frac{t}{\tau} $ }
\end{tabularx}

\subsection{RC}
\noindent\begin{tabularx}{\textwidth}{@{} 
p{\dimexpr.24\linewidth-2\tabcolsep-1.3333\arrayrulewidth} 
p{\dimexpr.26\linewidth-2\tabcolsep-1.3333\arrayrulewidth} 
p{\dimexpr.25\linewidth-2\tabcolsep-1.3333\arrayrulewidth} 
p{\dimexpr.25\linewidth-2\tabcolsep-1.3333\arrayrulewidth} @{}}
Impedance:                  & $ Z = \sqrt{X_C^2 + R^2} $
& Phase angle:              & $ \theta = arctan(\frac{X_C}{R}) $ \\
\multicolumn{4}{@{} l @{}}{Time Constant: $\tau = RC$ \qquad 1RC: 63\% \qquad 2RC: 86\% \qquad 3RC: 95\% \qquad 4RC: 98\% \qquad 5RC: 99\%} \\
\end{tabularx}

\subsection{Bipolar Junction Transistors}
\noindent\begin{tabularx}{\textwidth}{@{} 
p{\dimexpr.24\linewidth-2\tabcolsep-1.3333\arrayrulewidth} 
p{\dimexpr.26\linewidth-2\tabcolsep-1.3333\arrayrulewidth} 
p{\dimexpr.25\linewidth-2\tabcolsep-1.3333\arrayrulewidth} 
p{\dimexpr.25\linewidth-2\tabcolsep-1.3333\arrayrulewidth} @{}}
Emitter Voltage:            & $ V_E = V_B - 0.7v $
& Emitter Current:          & $ I_E = I_C + I_B $\\
Collector Current:          & $ I_C = I_B * \beta $
& (Int.) Emitter Resistance:& $ r_E = \frac{V_T}{I_E} \approx \frac{26mV}{I_E}$
\end{tabularx}

\subsection{Operational Amplifiers}
\noindent \begin{tabularx}{\textwidth}{@{} 
p{\dimexpr.24\linewidth-2\tabcolsep-1.3333\arrayrulewidth} 
p{\dimexpr.26\linewidth-2\tabcolsep-1.3333\arrayrulewidth} 
p{\dimexpr.25\linewidth-2\tabcolsep-1.3333\arrayrulewidth} 
p{\dimexpr.25\linewidth-2\tabcolsep-1.3333\arrayrulewidth} @{}}
Inverting amplifier:        & $ A = \frac{R_F}{R_I} $ 
& Non-inverting amplifier:  & $ A = \frac{R_F}{R_GND} + 1 $ \\
Buffer/Voltage follower:    & $ A = 1 $
& $\Rightarrow$             & $ v_{out} = v_{in} $
\end{tabularx}

\newpage
\begin{centering}
(This page was intentionally left blank.)\\
\end{centering}
\newpage

\section{Basic Concepts}\label{sec:basics}
\subsection{Water analogy}
Electronics and electricity is about the movement of electrons through 
different materials. The charge, voltage, current and resistance terms 
can be conceptually hard to understand. An analogy is often used, and the 
author prefers the \emph{water in pipes} analogy. Electrons or more generally 
charge can be thought of as water in a system of pipes.
\subsection{Charge}
Every electron has the same charge, called the elementary charge:
\[ e^- = -1.602*10^{-19} \text{C} \]
Electrons are defined to have negative charge. The absence of an electron, 
called a hole, is thought of as a charge carrier, even though it isn't a 
particle. Holes can move through the atomic structure of a material, much in 
the same way as electrons. Consequently a hole has precisely the opposite 
charge of an electron:
\[ e^+ = -e^- = 1.602*10^{-19} \text{C}\]
As electrons are negatively charged per definition it can often be useful to 
think of the movement of the positive charge carriers, the holes, instead. In 
the water analogy charge can be thought of as the water.
\subsection{Current}
Current, the flow of charge is defined as:
\begin{equation}
    I = \frac{Q}{t}
\end{equation}
Where \(Q\) is amount of charge and \(t\) is time. Thus current is the amount 
of charge passing a specific point in a unit of time (per second). In the 
water analogy current is simply a current of water, the rate at which water 
flows through a section of the pipe (liters / sec). The unit of current is 
Ampere:
\[ \text{A} = \frac{\text{C}}{\text{s}} \]
Ampere is a base SI Unit, meaning that Coulomb and the other units used in 
electronics are defined in terms of Ampere:
\[ \text{C} = 1 \text{A} * 1 \text{s} \]
The Coulomb is the summed charge of one Ampere for one second(the second is 
also an SI base unit).
\subsubsection{Measuring current}
When measuring current it is important to connect your multimeter/amperemeter 
in series with the point you want to measure. The current should flow through 
the meter. The meter should have very low impedance so it does not affect the 
circuit.
\subsection{Voltage}
Voltage is a measure of how much work is needed to move charge between two 
places. Thus, voltage is a relative, not absolute, unit. Voltage is a 
potential difference commonly from one point to a constant reference(ground). 
Definition:
\begin{equation}
    V = \frac{W}{Q}
\end{equation}
The voltage difference between points A and B is defined by how much 
work(energy, Joules) is required to move 1C of charge from A to B. In the 
water analogy voltage can be thought of as pressure in the pipes. It is easy 
to imagine the work required to push water from a pipe with lower pressure to 
a pipe with higher pressure. Furthermore voltage can create current. When completing a circuit current will flow from a place with higher voltage(+) to a place with lower voltage(-).
\subsubsection{Measuring Voltage}
When measuring voltage difference between two points in a circuit it is 
important to connect the multimeter/voltmeter in parallel. The meter will feel 
the difference in voltage between the two probes. High impedance through the 
meter is important. A meter with low impedance will draw current out of the 
circuit and change its function. See sections on resistance and impedance.

\subsection{Resistance}
A resistance limits the flow of current. We can easily imagine that longer pipes, smaller pipes, small openings or even bends in a pipe system can limit the flow of water. The same is true for current in wires. The formula which 
relates current through a resistance to applied voltage and resistance value is called Ohms law.
\subsubsection{Ohms law} 
The current through a resistance is proportional to the applied voltage and inversely proportional to the resistance value:
\begin{equation} \label{eq:ohm}
    I = \frac{V}{R} \Rightarrow V = R * I
\end{equation}
Impedance is a more general concept that includes resistance. Impedance is 
denoted with the \(Z\) character. Ohms law still holds:
\begin{equation}
    V = Z * I
\end{equation}
Whenever the book, this article or the lecture slides mention impedance, you 
can think it's like resistance. More on impedance later.
\subsubsection{Measuring resistance}
Measuring resistance is very similar to voltage, but it is important to 
disconnect the resistor from the rest of your circuit. The multimeter will 
send a small current through the resistor, so having other components 
connected can reduce the measured resistance.

\subsection{Voltage source}
A voltage source, for example a battery, applies a voltage difference to two 
points. An ideal voltage source will always deliver the desired voltage, regardless of how much current is drawn. Ideal voltage sources don't exist, 
but we can often assume that the voltage source is ideal (as long as we stay within a specified current range).

\subsection{Current source}
Similar to a voltage source the current source delivers a specified current 
regardless of required voltage. Current sources are not as common as voltage 
sources.

\subsection{Alternating and Direct Current(AC/DC)}
Voltage, current or any other signal can be varying in time(alternating) or 
constant over time(direct). This terminology is misleading as it usually has
nothing to do with current. In electronics the AC or DC terms are most 
commonly used about voltages. This terminology has been adopted in many 
different sciences, such as mathematics, statistics, physics and signal 
processing. 


\section{Power}\label{sec:power}
\subsection{Energy}
Energy is the ability to perform work. The unit, Joules, is expressed by:
\begin{equation}
    J = \frac{kg \times m^2}{s^2}
\end{equation}
In mechanics energy is usually thought of as either potential or kinetic, from 
gravity or motion respectfully. In electronics we usually talk about energy in 
terms of chemical potential energy in batteries or in terms of power.
\subsection{Power}
Power is the rate of energy, work per time:
\begin{equation}
    P = \frac{J}{s}
\end{equation}
\subsubsection{Watts Law}
When pushing current through a resistor or resistive circuit power is converted
to heat or light. The amount of power consumed is directly proportional to both 
voltage(pressure) and current(amount of water):
\begin{equation}
    P = V*I = \frac{V^2}{R} = I^2R
\end{equation}

\section{Impedance}
As mentioned earlier impedance is a more general form of \emph{resistance}. 
Impedance consists of resistance and reactance. This chapter focuses on the mathematical basis of impedance and related terms. Some concepts, like 
\emph{capacitors} and \emph{inductors}, are mentioned but not explained. These 
are explained in depth in separate chapters later

\subsection{Resistance}
Resistance limits the current flow. For a given voltage and resistance Ohm's law 
(\vref{eq:ohm}) can determine the current. Resistance does not vary with 
frequency. Resistors have resistance. Ideal resistors have only resistance, and 
no \emph{reactance}. In ideal resistors impedance should be constant for 
different voltages, temperatures and frequencies. Practical resistors are not 
perfect in these areas. The symbol for resistance is \(R\).
\subsection{Reactance}
Reactance can be found in capacitors and inductors, it is expressed in 
ohms(\(\Omega\)) just like resistance, but is different as it changes with 
frequency. The symbol for reactance is \(X\). For capacitive reactance 
(capacitors) \(X_C\) is commonly used, while \(X_L\) is commonly used for 
inductive reactance (inductors).
\subsection{Complex impedance}
Impedance is then a complex number, a sum of resistance and reactance:
\begin{equation}\label{eq:impedance}
    Z = R + jX
\end{equation}
For inductive reactance:
\begin{equation}\label{eq:impedanceind}
    Z = R + jX_L
\end{equation}
And for capacitive reactance:
\begin{equation}\label{eq:impedancecap}
    Z = R - jX_C
\end{equation}
The absolute value or magnitude of impedance is the length of the vector:
\begin{equation}
    |Z| = \sqrt{R^2+X^2}
\end{equation}
\subsection{Conductance}\label{sec:conductance}
Conductance is the opposite of resistance:
\begin{equation}
    G = \frac{1}{R}
\end{equation}
It quantifies how well a wire, component or material can 
\emph{conduct} electricity. Conductance is constant with respect to frequency. 
Conductance is more complicated for components which also have reactance:
\begin{equation}
    G = \frac{R}{|Z|^2}
\end{equation}
See \vref{sec:admittance} to understand why.
\subsection{Susceptance}\label{sec:susceptance}
Susceptance is the opposite of reactance:
\begin{equation}
    B = \frac{-1}{X}
\end{equation}
Just like conductance susceptance is more complicated for components which 
have both resistance and reactance:
\begin{equation}
    B = \frac{-X}{|Z|^2}
\end{equation}
Read \vref{sec:admittance} to better understand susceptance.
\subsection{Admittance}\label{sec:admittance}
Admittance is the opposite of impedance:
\begin{equation}
    Y = \frac{1}{Z}
\end{equation}
It is important to note that because of the imaginary unit it is not always 
trivial to convert between impedance and admittance:
\begin{equation}
    Y = \frac{1}{Z} = \frac{1}{R + jX} = 
    \frac{R-jX}{R^2+X^2} = \frac{R-jX}{|Z|^2}
\end{equation}
And for completeness, the difference between capacitors and inductors is shown:
\begin{align*}
    Y &= \frac{1}{Z} = \frac{1}{R - jX_C} = 
    \frac{R+jX_C}{R^2+X_C^2} = \frac{R+jX}{|Z|^2}\\
    Y &= \frac{1}{Z} = \frac{1}{R + jX_L} = 
    \frac{R-jX_L}{R^2+X_L^2} = \frac{R-jX}{|Z|^2}
\end{align*}
The minus sign in impedance for capacitive reactance can also be confusing.
In cases where an impedance is purely resistive or non-resistive this 
conversion is simpler:
\begin{align*}
    Y_R &= \frac{1}{Z_R} = \frac{1}{R} = G\\
    Y_C &= \frac{1}{Z_C} = \frac{1}{-jX_C} = j\frac{1}{X_C} = jB_C\\
    Y_L &= \frac{1}{Z_C} = \frac{1}{jX_L} = j\frac{1}{-X_L} = -jB_L\\
\end{align*}
\subsection{Updated Ohm's law}
With conductance and admittance we can express ohm's law in new, useful ways.
\begin{align*}
    V &= R*I, &\text{and}&&G &= \frac{1}{R}
\end{align*}
This is useful for expressing current through parallel resistances:
\begin{equation}\label{eq:ohmconductance}
    I = \frac{V}{R} \Rightarrow I = V*G
\end{equation}
Ohm's law also applies to impedance and admittance:
\begin{equation}\label{eq:ohmimpedance}
    V = R * I \Rightarrow V = Z * I
\end{equation}
\begin{equation}\label{eq:ohmadmittance}
    I = V * G \Rightarrow I = V * Y
\end{equation}
These formulae are easy to derive/remember and incredibly useful in some types 
of circuits.
\subsection{Impedance and Admittance table}
Below is a table that summarizes this section:\\
\newline
\begin{tabular}{| l | l | l |}
\hline
& Ohm, $\Omega$ & Siemens, S \\ \hline
Real(DC) & \textbf{Resistance}, R & \textbf{Conductance}, G \\
\hline
Imaginary(AC) & Capacitive and Inductive \textbf{Reactance} $X_C, X_L$ 
& \textbf{Susceptance}, B \\ 
\hline
Total(Combined)& \textbf{Impedance}, Z & \textbf{Admittance}, Y \\ 
\hline
\end{tabular}
\subsection{Impedance vector - magnitude and phase}

\section{Circuits}\label{sec:circuits}
\subsection{Fundamentals}
A circuit consists of \emph{nodes}, \emph{paths}, \emph{loops}, \emph{branches}
and \emph{components}. A \emph{component} can be a resistor, capacitor, 
battery, LED etc. \emph{Components} are connected by wires, a \emph{node} 
consists of all the wires that are connected together without any (significant) 
impedance. There is one \emph{node} on each pin of a \emph{component}. Pins or 
wires are in the same \emph{node} if they are directly connected, and thus 
have the same voltage. A \emph{path} is a road (through \emph{components}) 
where a current can flow. A \emph{loop} is a closed \emph{path} that connects 
to itself. A closed circuit has one or more \emph{loops} and \emph{paths}.

\subsection{Series circuits}
When connecting resistors in series we sum their resistance values to get the 
total:
\begin{equation}
    R_{TOT} = R_1 + R_2 + ... + R_n
\end{equation}
The same is true for non-resistive impedances.
\subsection{Parallel circuits}
When connecting resistors in parallel we sum their conductances to find total 
conductance:
\begin{equation}
    G_{TOT} = G_1 + G_2 + ... + G_n
\end{equation}
This makes sense; connecting two wires or equal resistors will double the 
conductance. We can easily find the resistance of two parallell resistors:
\begin{equation}
    R_{TOT} = \frac{1}{G_{TOT}} = \frac{1}{G_1 + G_2} = 
    \frac{1}{\frac{1}{R_1} + \frac{1}{R_2}}
\end{equation}
Or the common \emph{shortcut}:
\begin{equation}
    R_{TOT} = \frac{R_1R_2}{R_1+R_2}
\end{equation}
\subsection{Kirchhoffs Voltage Law}
The sum of all voltages going around a loop in a circuit is always 0.
\begin{equation}
    V_1 + V_2 + ... + V_n = 0
\end{equation}
This is because when we end up at the same node, we must also end up at the 
original voltage level.
\subsection{Kirchhoffs Current Law}
The sum of currents into/out of a node is always 0.
\begin{equation}
    I_1 + I_2 + ... + I_n = 0
\end{equation}
This is because there has to come as much charge out of as into a node. Beware that sign(direction) of current is important!
\subsection{Voltage divider}
A voltage divider takes an input voltage and produces a lower output voltage, 
that is a fraction of the input.
\begin{equation}
    V_x = V_S \frac{R_x}{R_{TOT}}
\end{equation}
Two or more resistors are used to split the voltage into different levels. The 
relative values of the resistors decide the voltages. For example: if one 
resistor is twice as big as the other, the voltage drop over it will be twice 
as big as the voltage drop over the other. A potentiometer is often used to 
create an adjustable output voltage.
\subsubsection{Load}
It is important to note that when connecting the voltage divider to a load the 
output voltage changes(!). Adding a load to the output will decrease the 
resistance down to ground and lower the voltage. A voltage divider with lower 
resistor values is better at supplying a lot of current to a load without 
changing output voltage (much).

\subsection{Current divider}
Current dividers are very similar but not as common as voltage dividers. 
When connecting resistors in parallel the current is divided among them. Recall 
the update ohm's law in \vref{eq:ohmconductance}:
\begin{equation*}
    I = V*G
\end{equation*}
Thus the current through one of the parallel resistors is:
\begin{equation*}
    I_X = V_{IN}*G_X
\end{equation*}
Total conductance is simply:
\begin{equation*}
    G_{TOT} = G_1 + G_2 + ... + G_n
\end{equation*}
And total current through all the resistors is:
\begin{equation*}
    I_{TOT} = V_{IN}*G_{TOT}
\end{equation*}
The more standard current divider formula is obtained by specifying input
current instead of voltage:
\begin{equation}
    I_X = (V_{IN})*G_X = (R_{TOT}*I_{IN})*G_X = \frac{G_X}{G_{TOT}}*I_{IN}
\end{equation}
Or in terms of resistance:
\begin{equation}\label{eq:currentdivider}
    I_X = \frac{R_{TOT}}{R_X} * I_{IN}
\end{equation}
It is crucial to remember that this is a parallel circuit, so:
\begin{equation*}
    R_X \geq R_{TOT}
\end{equation*}

\section{Periodic signals}\label{sec:signals}
\subsection{Sinusoid}
The sinusoid waveform or \textit{sine wave} is the fundamental type of alternating current (ac) or alternating voltage. It is also reffered to as a sinusoidal wave, or simply sinusoid. In addition, other types of repetitive \textit{waveforms} are composites of many individual sine waves called harmonics.
\newline \textit{Sinusoids} are produced by two types of sources: \begin{itemize}
    \item Rotating electrical machines (ac generator)
    \item Electronic oscillator circuits, which are used in instruments commonly known as electronic signal generators
\end{itemize} 
\subsubsection{Amplitude}
The maximum value of a voltage or current.

\subsubsection{Frequency \& Period}
\textit{Frequency} is the number of cycles that a sinusoid completes in one second. The more cycles completed in one second, the higher the frequency. Frequency (f) is measured in unites of hertz (Hz).
\newline 

A \textit{sinusoid} varies with time (\textit{t}) in a definable manner. The time required for a given sinusoid to complete one full cycle is called the period(\textit{T}). 

The formulas for the relationship between frequency and period are
\[f = \frac{1}{T}\] 

\[T = \frac{1}{f}\]


\subsubsection{Phase shift}
The phase of a sinusoid is an angular measurement that specifies the position of that sinusoid relative to a reference. When a sinusoid is shifted to the right of the reference (lags) by a certain angle, \textbf{$\phi$}. 

\[y = A\, sin(\theta-\phi)\] 

\subsection{Square wave}

A \textit{Square wave} is a pulse waveform with a duty cycle of 50\%.
\subsubsection{Pulse width \& Duty Cycle}
\textit{Pulse width} is the time between the point on the leading edge where the value is 50 \% of the amplitued and the point on the trailing edge where the value is 50\% of the amplitude. 

\subsubsection{Rise \& fall time}
\textit{Rise time} is the time required for the pulse to go from 10\% of its amplitued to 90\% of its amplitude. 
\newline \textit{Fall time} is the time required for the pulse to go from 90\% of its amplitued to 10\% of its amplitude.


\section{Capacitors}\label{sec:cap}
\subsection{Physical properties}
\subsection{Stored capacitor charge}
\subsection{Time constant}
\subsection{Capacitive reactance}
\subsection{Voltage \& Current}
\subsection{Parallell capacitors}
\subsection{Series capacitors}
\subsection{Practical capacitors}
\subsection{RC circuits}
\subsection{Capacitive voltage divider}
\subsection{Power consumption in capacitors}

\newpage

\section{Inductors}\label{sec:ind}
\subsection{Physical properties}
\subsection{Induced voltage}
\subsection{Inductive reactance}
\subsection{Practical Inductors}
\subsection{RL circuits}


\section{Filters and frequency Response}\label{sec:frequency}
\subsection{RLC characteristics}
$X_L$ and $X_C$ have opposing effects in an RLC circuit. In a series RLC circuit, the larger reactance determines the net reactance of the circuit. In a parallel RLC cicuit, the smaller reactance determines the net reactance of the circuit. 

\subsection{Time domain}


\subsection{Frequency domain}
In electric circuits, the variation in the output voltage (or current) over a specified range of frequencies.

\subsection{Harmonics}
A repetitive nonsinusoidal waveform contains sinusoidal waveforms with a fundamental frequency and harmonic frequencies. The fundamental frequency is the repetition rate of the waveform, and the harmonics are higher-frequency sine waves that are multiples of the fundamental. 
\begin{itemize}
    \item Odd Harmonics are frequencies that are odd multiples of the fundamental frequency of a waveform.
    
    \item Even Harmonics are frequencies that are even multiples of the fundamental frequency. 
\end{itemize}
Composite waveform is any variation from a pure sine wave which produces harmonics. 

\subsection{Fourier series and Fourier transform}


\subsection{Low-Pass filter}
A low-pass filter is used to only let the low frequencies pass through the circuit. 

\subsection{High-Pass filter}
A high-pass filter is used to only the the high frequencies pass through the circuit. 

\subsection{Band-pass filter}
A \emph{band-pass filter} allows signals at the resonant frequency and at frequencies within a certain band (or range) extending below and above the resonant value to pass from input to output without a significant reduction in amplitude. Signals at frequencies lying outside this specified band (called the \emph{passband}) are reduced in amplitude to below a certain level and are considered to be rejected by the filter.
The formula for calculating the bandwidth is 
\[BW = f_2 - f_1\]

\subsection{Band-stop filter}
The \emph{band-stop filter} rejects signals with frequencies between the lower and upper cutoff frequencies and passes those signals with frequencies below and above the cutoff values. The range of frequencies between the lower and upper cutoff points is called the stopband. This type of filter is also referred to as a \emph{band-elimination filter}, \emph{band-reject filter}, or a \emph{notch filter}. 


\section{Resources / Further Reading}\label{sec:resources}
This Introduction to Analog Electronics was inspired by previous works by 
different authors. The list below should serve as a recommended reading list 
for anyone who wants to learn more.
\begin{itemize}
    \item "Electronics Fundamentals: Circuits, Devices and Applications"
    \cite{fundamentals}
    \item "INF1411 Formelsamling" \cite{hagenes}
    \item "Formelsamling i Elektronikk" \cite{formulae} 
    \item "Analog Integrated Cicruit Design" \cite{analog}
\end{itemize}

\printbibliography

\end{document}
% ==============================================================================
