As mentioned earlier impedance is a more general form of \emph{resistance}. 
Impedance consists of resistance and reactance. This chapter focuses on the mathematical basis of impedance and related terms. Some concepts, like 
\emph{capacitors} and \emph{inductors}, are mentioned but not explained. These 
are explained in depth in separate chapters later

\subsection{Resistance}
Resistance limits the current flow. For a given voltage and resistance Ohm's law 
(\vref{eq:ohm}) can determine the current. Resistance does not vary with 
frequency. Resistors have resistance. Ideal resistors have only resistance, and 
no \emph{reactance}. In ideal resistors impedance should be constant for 
different voltages, temperatures and frequencies. Practical resistors are not 
perfect in these areas. The symbol for resistance is \(R\).
\subsection{Reactance}
Reactance can be found in capacitors and inductors, it is expressed in 
ohms(\(\Omega\)) just like resistance, but is different as it changes with 
frequency. The symbol for reactance is \(X\). For capacitive reactance 
(capacitors) \(X_C\) is commonly used, while \(X_L\) is commonly used for 
inductive reactance (inductors).
\subsection{Complex impedance}
Impedance is then a complex number, a sum of resistance and reactance:
\begin{equation}\label{eq:impedance}
    Z = R + jX
\end{equation}
For inductive reactance:
\begin{equation}\label{eq:impedanceind}
    Z = R + jX_L
\end{equation}
And for capacitive reactance:
\begin{equation}\label{eq:impedancecap}
    Z = R - jX_C
\end{equation}
The absolute value or magnitude of impedance is the length of the vector:
\begin{equation}
    |Z| = \sqrt{R^2+X^2}
\end{equation}
\subsection{Conductance}\label{sec:conductance}
Conductance is the opposite of resistance:
\begin{equation}
    G = \frac{1}{R}
\end{equation}
It quantifies how well a wire, component or material can 
\emph{conduct} electricity. Conductance is constant with respect to frequency. 
Conductance is more complicated for components which also have reactance:
\begin{equation}
    G = \frac{R}{|Z|^2}
\end{equation}
See \vref{sec:admittance} to understand why.
\subsection{Susceptance}\label{sec:susceptance}
Susceptance is the opposite of reactance:
\begin{equation}
    B = \frac{-1}{X}
\end{equation}
Just like conductance susceptance is more complicated for components which 
have both resistance and reactance:
\begin{equation}
    B = \frac{-X}{|Z|^2}
\end{equation}
Read \vref{sec:admittance} to better understand susceptance.
\subsection{Admittance}\label{sec:admittance}
Admittance is the opposite of impedance:
\begin{equation}
    Y = \frac{1}{Z}
\end{equation}
It is important to note that because of the imaginary unit it is not always 
trivial to convert between impedance and admittance:
\begin{equation}
    Y = \frac{1}{Z} = \frac{1}{R + jX} = 
    \frac{R-jX}{R^2+X^2} = \frac{R-jX}{|Z|^2}
\end{equation}
And for completeness, the difference between capacitors and inductors is shown:
\begin{align*}
    Y &= \frac{1}{Z} = \frac{1}{R - jX_C} = 
    \frac{R+jX_C}{R^2+X_C^2} = \frac{R+jX}{|Z|^2}\\
    Y &= \frac{1}{Z} = \frac{1}{R + jX_L} = 
    \frac{R-jX_L}{R^2+X_L^2} = \frac{R-jX}{|Z|^2}
\end{align*}
The minus sign in impedance for capacitive reactance can also be confusing.
In cases where an impedance is purely resistive or non-resistive this 
conversion is simpler:
\begin{align*}
    Y_R &= \frac{1}{Z_R} = \frac{1}{R} = G\\
    Y_C &= \frac{1}{Z_C} = \frac{1}{-jX_C} = j\frac{1}{X_C} = jB_C\\
    Y_L &= \frac{1}{Z_C} = \frac{1}{jX_L} = j\frac{1}{-X_L} = -jB_L\\
\end{align*}
\subsection{Updated Ohm's law}
With conductance and admittance we can express ohm's law in new, useful ways.
\begin{align*}
    V &= R*I, &\text{and}&&G &= \frac{1}{R}
\end{align*}
This is useful for expressing current through parallel resistances:
\begin{equation}\label{eq:ohmconductance}
    I = \frac{V}{R} \Rightarrow I = V*G
\end{equation}
Ohm's law also applies to impedance and admittance:
\begin{equation}\label{eq:ohmimpedance}
    V = R * I \Rightarrow V = Z * I
\end{equation}
\begin{equation}\label{eq:ohmadmittance}
    I = V * G \Rightarrow I = V * Y
\end{equation}
These formulae are easy to derive/remember and incredibly useful in some types 
of circuits.
\subsection{Impedance and Admittance table}
Below is a table that summarizes this section:\\
\newline
\begin{tabular}{| l | l | l |}
\hline
& Ohm, $\Omega$ & Siemens, S \\ \hline
Real(DC) & \textbf{Resistance}, R & \textbf{Conductance}, G \\
\hline
Imaginary(AC) & Capacitive and Inductive \textbf{Reactance} $X_C, X_L$ 
& \textbf{Susceptance}, B \\ 
\hline
Total(Combined)& \textbf{Impedance}, Z & \textbf{Admittance}, Y \\ 
\hline
\end{tabular}
\subsection{Impedance vector - magnitude and phase}