\subsection{Water analogy}
Electronics and electricity is about the movement of electrons through 
different materials. The charge, voltage, current and resistance terms 
can be conceptually hard to understand. An analogy is often used, and the 
author prefers the \emph{water in pipes} analogy. Electrons or more generally 
charge can be thought of as water in a system of pipes.
\subsection{Charge}
Every electron has the same charge, called the elementary charge:
\[ e^- = -1.602*10^{-19} \text{C} \]
Electrons are defined to have negative charge. The absence of an electron, 
called a hole, is thought of as a charge carrier, even though it isn't a 
particle. Holes can move through the atomic structure of a material, much in 
the same way as electrons. Consequently a hole has precisely the opposite 
charge of an electron:
\[ e^+ = -e^- = 1.602*10^{-19} \text{C}\]
As electrons are negatively charged per definition it can often be useful to 
think of the movement of the positive charge carriers, the holes, instead. In 
the water analogy charge can be thought of as the water.
\subsection{Current}
Current, the flow of charge is defined as:
\begin{equation}
    I = \frac{Q}{t}
\end{equation}
Where \(Q\) is amount of charge and \(t\) is time. Thus current is the amount 
of charge passing a specific point in a unit of time (per second). In the 
water analogy current is simply a current of water, the rate at which water 
flows through a section of the pipe (liters / sec). The unit of current is 
Ampere:
\[ \text{A} = \frac{\text{C}}{\text{s}} \]
Ampere is a base SI Unit, meaning that Coulomb and the other units used in 
electronics are defined in terms of Ampere:
\[ \text{C} = 1 \text{A} * 1 \text{s} \]
The Coulomb is the summed charge of one Ampere for one second(the second is 
also an SI base unit).
\subsubsection{Measuring current}
When measuring current it is important to connect your multimeter/amperemeter 
in series with the point you want to measure. The current should flow through 
the meter. The meter should have very low impedance so it does not affect the 
circuit.
\subsection{Voltage}
Voltage is a measure of how much work is needed to move charge between two 
places. Thus, voltage is a relative, not absolute, unit. Voltage is a 
potential difference commonly from one point to a constant reference(ground). 
Definition:
\begin{equation}
    V = \frac{W}{Q}
\end{equation}
The voltage difference between points A and B is defined by how much 
work(energy, Joules) is required to move 1C of charge from A to B. In the 
water analogy voltage can be thought of as pressure in the pipes. It is easy 
to imagine the work required to push water from a pipe with lower pressure to 
a pipe with higher pressure. Furthermore voltage can create current. When completing a circuit current will flow from a place with higher voltage(+) to a place with lower voltage(-).
\subsubsection{Measuring Voltage}
When measuring voltage difference between two points in a circuit it is 
important to connect the multimeter/voltmeter in parallel. The meter will feel 
the difference in voltage between the two probes. High impedance through the 
meter is important. A meter with low impedance will draw current out of the 
circuit and change its function. See sections on resistance and impedance.

\subsection{Resistance}
A resistance limits the flow of current. We can easily imagine that longer pipes, smaller pipes, small openings or even bends in a pipe system can limit the flow of water. The same is true for current in wires. The formula which 
relates current through a resistance to applied voltage and resistance value is called Ohms law.
\subsubsection{Ohms law} 
The current through a resistance is proportional to the applied voltage and inversely proportional to the resistance value:
\begin{equation} \label{eq:ohm}
    I = \frac{V}{R} \Rightarrow V = R * I
\end{equation}
Impedance is a more general concept that includes resistance. Impedance is 
denoted with the \(Z\) character. Ohms law still holds:
\begin{equation}
    V = Z * I
\end{equation}
Whenever the book, this article or the lecture slides mention impedance, you 
can think it's like resistance. More on impedance later.
\subsubsection{Measuring resistance}
Measuring resistance is very similar to voltage, but it is important to 
disconnect the resistor from the rest of your circuit. The multimeter will 
send a small current through the resistor, so having other components 
connected can reduce the measured resistance.

\subsection{Voltage source}
A voltage source, for example a battery, applies a voltage difference to two 
points. An ideal voltage source will always deliver the desired voltage, regardless of how much current is drawn. Ideal voltage sources don't exist, 
but we can often assume that the voltage source is ideal (as long as we stay within a specified current range).

\subsection{Current source}
Similar to a voltage source the current source delivers a specified current 
regardless of required voltage. Current sources are not as common as voltage 
sources.

\subsection{Alternating and Direct Current(AC/DC)}
Voltage, current or any other signal can be varying in time(alternating) or 
constant over time(direct). This terminology is misleading as it usually has
nothing to do with current. In electronics the AC or DC terms are most 
commonly used about voltages. This terminology has been adopted in many 
different sciences, such as mathematics, statistics, physics and signal 
processing. 
